
\documentclass[%
 reprint,
%superscriptaddress,
%groupedaddress,
%unsortedaddress,
%runinaddress,
%frontmatterverbose, 
%preprint,
%showpacs,preprintnumbers,
%nofootinbib,
%nobibnotes,
%bibnotes,
 amsmath,amssymb,
 aps,
%pra,
%prb,
%rmp,
%prstab,
%prstper,
%floatfix,
]{revtex4-1}

\usepackage{graphicx}% Include figure files
\usepackage{dcolumn}% Align table columns on decimal point
\usepackage{bm}% bold math
\usepackage{mathabx}
\usepackage{textcomp}
\usepackage{amsmath,amssymb}
%\usepackage{hyperref}% add hypertext capabilities
%\usepackage[mathlines]{lineno}% Enable numbering of text and display math
%\linenumbers\relax % Commence numbering lines

%\usepackage[showframe,%Uncomment any one of the following lines to test 
%%scale=0.7, marginratio={1:1, 2:3}, ignoreall,% default settings
%%text={7in,10in},centering,
%%margin=1.5in,
%%total={6.5in,8.75in}, top=1.2in, left=0.9in, includefoot,
%%height=10in,a5paper,hmargin={3cm,0.8in},
%]{geometry}

%AIP
%%%%%%%%%%%%%%%%%%%%%%%%%%%%%%%%%%%%%%%%%%%%%%%


%\documentclass{article}
\usepackage[utf8]{inputenc}
\usepackage{mathtools}
\begin{document}
\title{Scattering of the Higgs Field off of Quantum Gravity Topological Fluctuations}

\author{Evin O'Shea}
 \affiliation{Physics Department, University of New Hampshire}
\date{May 2018}



\begin{abstract}

Any work of communication should include a focus on the audience of the piece. In this case, the presenter is not more knowledgable than the audience. This means that the presentation will potentially include blatant errors. This is difficult to combat, however, the learning experience of attempting to formulate a description of concepts that are very unfamiliar and are at a level above the presente's knowledge is still useful. Here, I attempt to outline the results of qualitative calculations of scattering of scalar fields off of quantum gravity topological fluctuations. I then look to interpret the work that was done before the discovery of the Higgs Boson (a scalar field) into an interpretation of these calculations given the modern knowledge of the Higgs boson.

\end{abstract}



\maketitle

%\begin{center}
\newpage

\tableofcontents

\newpage
%\end{center}

% \section{Please Remove This Section}

% Minihood1997 17 points 2 years ago 

% Photons are the gauge bosons of Electromagnetism. So how can the charge of a black hole exist, if the electric force caused by it cannot be felt outside the event horizon since photons cannot escape it? Or am I completely misunderstanding gauge bosons?



% Why are all spectra states transitionable by light?


% Yukawa potential 


% It is important to emphasize that, although (2.12) is similar in appearance to the
% familiar position–momentum uncertainty relation, its origin and interpretation is
% quite different. In the case of x and p, both quantities are dynamical variables,
% whose incompatibility (non-commutativity) leads directly to the result $\Delta x \Delta p \geq \hbar/2$. However, although ‘energy’ is certainly a dynamical variable, ‘time’ is not: it is simply a parameter. There is no operator, conjugate to the Hamiltonian,
% whose eigenvalue is ‘the time’ (see, for example, Gottfried 1966, p 248).


% \

% \begin{figrue}[h]
% \includegraphics[scale=0.35]{QCD.png}
% \caption{The Quantum Chromodynamic Largrangian. This largrangian is used th formulate the Hamiltonian for these fields and thereby compute scattering amplitudes. It is a lot easier to state the process than to do the calculations. \cite{qcd} }
% \end{figure}





% Analytic continuation -



% Since we cann
% ot perform this path
% integral, we then restricted the discussion to one such metr
% ic, and analytically continued
% the solution to a Lorentzian section to make the scattering e
% asier to understand."


% "The fact that there is a non-zero number of particles at I + implies that there is loss of quantum coherence in this semi-classical calculation"



% "On the other hand, everyday observations indicate that spacetime is nearly flat when viewed on normal scales."

% "It also suggests that the scattering of higher-spin fields off su ch virtual black hole loops will be suppressed relative to that of scalar fields, as they c annot radiate in the s-wave. This is in agreement with the arguments of [20, 19]."


% "The reason is that the action for the gravitational field is not scale invariant, unlike that of the Yang-Mills or electromagnetic fields. This means that a large fluctuation of the metric over a short length scale does not have a very large action."

% "However, there’s a very definite discontinuity when event horizons form: the Euclidean topology of spacetime will change from that of flat space, to something non-trivial. The change in topology will mean that any vector field that agrees with time translations at infinity, will necessarily have zeroes in the interior of the spacetime."

% "Particles could fall into these virtual black holes, which would then radiate other particles. The emitted particles would be in a mixed quantum state because the presence of the black hole horizons will mean that a quantum state will not be determined completely by its behavior at infinity."

% "On the manifold S 2 × S 2 − {point} one should consider Euclidean metrics that are asymptotic to flat space at infinity. Such metrics can be interpreted as closed loops of virtual black holes. Because they are off shell, they need not satisfy any field equations. They will contribute to the path integral, just as off shell loops of particles contribute to the path integral and produce measurable effects. The effect that we shall be concerned with for virtual black holes is loss of quantum coherence. This is a distinctive feature of such topological fluctuations that distinguishes them from ordinary unitary scattering, which is produced by fluctuations that do not change the topology."

% "As explained earlier, the Hilbert space is
% isomorphic to the tensor product of the Fock spaces on the two horizons (1). Positive
% frequency on the Fock spaces is defined with respect to the affine parameter along the
% horizon. The state we wish to study is the analytically-continued Euclidean vacuum
% state |0i E given in (2)."

% "We consider a minimally-coupled massless neutral scalar field, so the wave equation is just ✷φ = 0."

% "The C metric is a solution of the vacuum Einstein-Maxwell equations"

% "But in solving this latter equation, we are free to make conformal transformations. In particular, we can transform to the conformal gauge"

% "The basic properties of the C-metric are well known. It describes a pair of causally separated black holes which accelerate in opposite directions under the action of forces represented by conical singularities. However, these properties can be demonstrated much more transparently by making use of recently developed coordinate systems for which the metric functions have a simple factor structure"

\section{Introduction}

An important part of quantum gravity research is calculating the interactions between known fields and gravity in the theory. Quantum gravity would include (at least in principle) a way of integrating over virtual quantum gravity "particles". In the papers to be discussed the quanta of the theory are thought of as topological fluctuations of the metric. Stephen Hawking was an author on several papers that investigate different topological fluctuations such as "virtual black hole loops" \cite{virtualblackholes} and "spacetime foam" \cite{spacetimefoam}. This work has been to investigate the scattering of scalar fields off of these topological fluctuations \cite{virtualblackholes} \cite{virtualfoam}. The results of these investigations led Stephen Hawking think the Higgs boson would not be discovered or that it would not be a fundamental particle (it would be made of multiple elementary spin $\>$ 0 particles). Given the measurements of the Higgs boson \cite{higgsboson} which have led us to believe that it is a fundamental particle, may instead imply that the topological fluctuations that Hawking has described are not physical. These calculations are done with particular metrics such as the done with the C metric ("virtual black holes"). These calculations can give qualitative results that we may expect to indicate results obtained from properly integrating the Greens functions over all possible metrics. If topological fluctuations are the correct way to formulate quantum gravity it would mean that arbitrary calculations would not be feasible. Physicist hope that calculations can be done in the correct theory of quantum gravity. Therefore, it would be very enlightening if these calculations indicate problems with fromulating quantum gravity as these topological fluctuations.

\section{Quantum Field Theory}
A field is an object (usually a linear operator) that is defined at every point in spacetime. In the framework we define the field as a combination of the creation and annihilation operators \cite{atchandhey} 

\begin{equation}
\phi(\vec{x}) = \int \frac{d^3 p}{(2\pi)^3} \frac{1}{\sqrt{2\omega_{\vec{p}}}} \left[ a_{\vec{p}} \, e^{ i \vec{p} \cdot \vec{x}} + a_{\vec{p}} ^{\dagger} \, e^{ - i \vec{p} \cdot \vec{x}} \right]
\end{equation}

interpreted as raiseing and lowering operators for the number of quanta (or particles) of the field. The way to use this theory for calculations is to calculate the Hamiltonian and assert the usual commutation relationship:

\begin{equation}
\left[ \phi(\vec{x}) , \, \pi(\vec{y}) \right] =  i \hbar \, \delta(\vec{x} - \vec{y})
\end{equation}

know as ``cannonical quantization" \cite{tong}. The Hamiltonian for the system istaken to be the integral over all number operators:

\begin{equation}
H = \int d^4p \,\, a_{\vec{p}} ^{\dagger} \, a_{\vec{p}}
\end{equation}

Where we have neglected the other contributions to the Hamiltonian (this is how we define the ground state \cite{atchandhey}). The Hamiltonian is the total energy of the system. Therefore the Hamiltonian is the number of particles times the energy per particle, which is the total energy. 

In quantum field theory we use this framework and formulate our fields as combinations of fundamental particles that are described by operators that represent creation and anihilation of that particle type. In quantum gravity we look for an analogy that applies to "fluctuations" relating to the topology of metric.

\section{The Higgs Boson}
The Higgs boson is a fundamental scalar field with mass \raise.17ex\hbox{$\scriptstyle\sim$}126GeV \cite{higgsboson}. It is responsible for the emass of the electroweak bosons. The measurement of the Higgs boson informs us of constrants on the interactions of scalar fields in quantum gravity theories. In particular a quantum gravity result that would cause scalar fields to have a mass greater than the mass of the Higgs boson would be ruled out as acceptable theories.

\section{Conformal Transformations}
Conformal transformations are functions of coordinates that are scalar multiplied by the metric to obtain a new metric:

\begin{equation}
g_{\mu \nu}' = \Omega^2 (x) \, g_{\mu \nu}
\end{equation}

There is investigation into conformal field theories in which symmetries under continuous conformal transformations determine the associated gauge fields. This is similar to the standard model approach. Here, the associated gauge fields are considered to be the graviational fluctuations.

Conformal field theories face immense difficulties because there is an infinite-demensional algrbra of such local conformal transformations. 

\section{Metric Fluctuations}

The term metric fluctuations refers to the contribution to the quantum gravity path integral of the integral over all metrics of the Greens functions (the fluctuations, analogous to standard model ``particles", thought of as ``spacetime foam", ``virtual black hole pairs") for each metric. This infinite-dimensional integral is not feasbile, but it is possible to demonstrate qualitative results using the C metric \cite{virtualblackholes} (or other metrics \cite{virtualfoam}) and the associated field solutions.

\section{Greens Functions for Topological Fluctuations}

The Euclidean Greens Functions are the set of metric topologies that would be integrated over in a path integral formulation of quantum gravity. Integrating over all metrics in this case is analogous to the integral over all paths in QFT.

\begin{equation}
H = \int d^4 x \mathcal{H} 
\end{equation}

There is a quantum state obtained from the Euclidean green functions by this integral. This can be done for certain metrics with field solutions such as the C metric. This would be one contribution to the path integral.

%``The Euclidean section has topology S 2 × S 2 − {pt}. This Euclidean section can be used to study the pair creation of black holes by breaking cosmic strings [15, 16, 17]. However, we want to use it simply to determine the appropriate vacuum state on the Lorentzian section. Since the black hole and acceleration horizon have the same temperature on the Euclidean section, the analytic continuation will give Green’s functions which are thermal with temperature 1/β with respect to the time parameter t in the Lorentzian section"

\section{Scattering Off Virtual Black Holes}

In \cite{virtualblackholes} a calculation was done with the Euclidian C metric (C metric), which has a toplogy of $S^2 \times S^2 - \{point\}$. The calculation for scattering of scalar fields off virtual black hole loops uses a finite dimensional approximation to an asymptotically Euclidean metric, in this case the C metric. The C metric is a solution of the conformal field equation.

The result of the calculation is a non-zero number of particles at infinity which is a loss of quantum coherence (a violation of unitarity) in the qualitative ``semi-classical calculation". The calculation obtains a transmission factor such that the dominant contribution to the particle production is in the s-wave which is surpressed for scattering of higher-spin fields off such virtual black hole loops relative to that of scalar fields because scalar fields cannot radiate in the s\textemdash wave.

This calculation led the authors to the conclusion that fundamental scalar fields would not exist. The measurement of the Higgs Boson would therefore imply that if the conclusions from the calculation are correct, these topological fluctuations would not be physical as they would violate unitarity.

\section{Scattering Off Spacetime Foam}

The ``gravitational bubbles" of the ``spacetime foam" can be thouhgt of as being built out of three units of ``topological fluctuation", $CP^2$, $S^2 \times S^2$ and K3. The results of a qualitative analysis demonstrated for the ``quantum gravity bubbles" \cite{qgbubbles} associated with the ``spacetime foam" is that the scattering amplitudes are very small for fermions and vectors but would predict scalar particles to have a mass on the order of the Plank mass ($M_P \raise.17ex\hbox{$\scriptstyle\sim$} 1.2 \times 10^{19}$ GeV) \cite{virtualfoam}.

The scattering amplitudes of the fluctuations investigated in \cite{virtualfoam} are of order unity for scalar particles and could give rise to an effective mass of the order of the Planck mass; however, fermions and vectors will not acquire a mass.

This calculation led the authors to a similar conclusion to the case of virtual black hole loops, which is that scaar fields either would not be fundamental, would have a large mass, or would not be discovered at all. This concusion is fundamentally incorrect given the measurements of the Higgs boson.


\section{Discussion}

Since the Higgs Boson has been discovered and its measurements have indicated it is a fundmental particle, this may mean that the topological fluctuations proposed do not exist. It may be nice if nature is not a quanum gravity theory of topological fluctuations of the metric since there is no way to do the infinite\textemdash dimentional integral over all metric fluctuations.

Small successes that add to the constraints of our physical models are important. Though it is not clear if these qualitative calculations are indicative of the results of a proper conformal field theory calculation, it they are, then the topological fluctuations proposed such as virtual black hole loops and gravitational bubbles, could not be physically correct.

\begin{thebibliography}{8}

\bibitem{virtualblackholes} S. W. Hawking, Simon F. Ross. Loss of quantum coherence through scattering off virtual black holes. DAMTP-R-97-21, UCSB-TH-97-08. May 1997. 29 pp. Published in Phys.Rev. D56 (1997) 6403-6415. arXiv:hep-th/9705147, DOI: 10.1103/PhysRevD.56.6403 

\bibitem{spacetimefoam} S. W. Hawking, Spacetime foam, Nuclear Physics B, Volume 144, Issues 2–3, 1978, Pages 349-362, ISSN 0550-3213, https://doi.org/10.1016/0550-3213(78)90375-9.

\bibitem{virtualfoam} S.W. Hawking, D.N. Page, C.N. Pope, The propagation of particles in spacetime foam, Physics Letters B, Volume 86, Issue 2, 1979, Pages 175-178, ISSN 0370-2693, https:// doi.org/10.1016/0370-2693(79)90812-8.

\bibitem{higgsboson} Observation of a new particle in the search for the standard model Higgs boson with the ATLAS detector at the LHC, Physics Letters B, Volume 716, Issue 1, 2012, Pages 1-29, ISSN 0370-2693, https://doi.org/10.1016/j.physletb.2012.08.020.\

\bibitem{atchandhey} I. J. R. Aitchison (Author), A. J. G. Hey (2003), Gauge Theories in Particle Physics, Vol. 1: From Relativistic Quantum Mechanics to QED, 3rd Edition. United Kingdom, Taylor \& Francis Group, ISBN-13: 978-0750308649.

\bibitem{tong} David Tong, Lectures on Quantum Field Theory. University of Cambride. Online: http://www.damtp.cam.ac.uk/user/tong/qft.html.

\bibitem{cmetric} J. B. Griffiths, P. Krtous, J. Podolsky (2006), Interpreting the C-metric, Class. Quant. Grav. 23 6745-6766. arXiv:gr-qc/0609056.

\bibitem{qgbubbles} S.W. Hawking, D.N. Page, C.N. Pope, Quantum gravitational bubbles, Nuclear Physics B, Volume 170, Issue 2, 1980, Pages 283-306, ISSN 0550-3213, https://doi.org/10.1016/0550-3213(80)90151-0.

\end{thebibliography}

\end{document}