\documentclass[12pt,epsfig]{article}

  \usepackage{graphicx}
  \usepackage{hyperref}



\begin{document}


\title{Society of Physics Students Standard Operating Procedures}
\author{Evin O'Shea}

\maketitle

\abstract{This document will outline the Standard Operating Procedures of SPS. This guide will help future organization leaders through their time in SPS. This document will help structure the leadership roles as well as outline how operations work. This document should be updated when necessary and should be used as the tool it is. This document should be updated by officers at the end of each semester.}



\newpage

\pagestyle{plain}

\tableofcontents

\newpage

\section{Goals}
First, the organization must have clear goals. These goals can be revisited as the may be specific semester to semester.
\begin{itemize}
	\item Engage the students in the physics department. Provide free and fun social activities to students.

	\item Educate students and be a resource for students looking for information and advice

	\item An organization that offers tutoring for the introductory (PHYS 401/402/407/408) physics classes.
\end{itemize}

\section{Re-recognition}
There is a MANDATORY student org. meeting that at least one member from each org. must attend and sign that they were there. If this does not happen, the organization may be un-recognized or will be made to jump through many fiery hoops before they are left alone. After attending the meeting, an org. leader (Should be the President) must fill out the re-recognition form to ``apply" for re-recognition. By attending the meeting and signing, and filling out this form you will prevent a lot of unnecessary difficulties for the org.

\section{Members}
There are not and should not be requirements of any member that is not in a leadership role and has not volunteered themselves for a specific tasks (one should always follow through on tasks they volunteered to do). Members are welcome to take on more responsibilities, but should never be pressured. The org. is meant to bring the physics department together and the people who are heavily active are being  

\section{Event Operating procedures}
This is not how it must work, this is a way that it works to create and host and event as an org. Some of these points will be obviously mandatory and some will be suggestions.

\begin{itemize}
	\item First, an event idea has to come to fruition, usually at a regular meeting or at least it is agreed upon at a meeting. 

	\item After the idea for the event is set, purchases need to be determined. Then, to make a purchase, SOMEONE must request a P-card, use it, and fill out a receipt form. (See~\ref{Using A P-Card}) 

	\item Each event needs a time and location. The large lecture hall (112) is controlled by the registrar's office and most of the rest of the rooms in Demerit are controlled by Katie. A room must be reserved for the event either in person or through email to Katie or through the registrar's office: \url{http://www.unh.edu/registrar/general-information/room-request/}

	\item Lastly, any event should be advertised for. This works best through word of mouth if it is spread well. Other possibilities include announcements to classes, notes left on the chalkboards and an email to members or through Katie (this could be sent to all physics students).
\end{itemize}

\section{Finance}
This section will be dedicated to organization finance. This section will explain the importance of tutoring as a source of income, budgeting, and using a P-Card

\subsection{Using A P-Card} \label{Using A P-Card}
Using a P-Card is simple, but intimidating at first. With this resource it should be a simple task. It is important to remember that the people in the OSIL office are there to help you. Make sure to tell them what you are there for and they will help you as much as you need.

\begin{itemize}
	\item First, log on to \url{https://wildcatlink.unh.edu/}. Then, go to the top right search button, search for ``organization finance", and select the organization "Organization Finance - Memorial Union \& Student Activities". Go to the ``Forms" tab and find the form ``Student Organization P-Card Request Form 20XX - 20X(X+1)". Fill out this form and submit it. The organization account number can be found on the Student Organization Finance Sheet (which can be found on the wildcatlink page under the Documents tab with the title ``Society of Physics Students Account Info.pdf")

	\item After the P-Card is requested, the OSIL office will eventually accept your request for a P-Card. Then you can go and pick it up in the OSIL office (in the MUB). You have 3 days to return your P-Card. If you are just ordering pizza, you can order in the OSIL office and then you can immediately return the P-Card. It is okay to use the card with multiple vendors as long as you state them all.

	\item The next step is the simplest but maybe the most important: save the receipt.

	\item Once you have the receipt, bring it to the OSIL office and fill out a Receipt Form. If you lose the receipt you have to fill out a lost receipt form which is worse, so just keep the receipt. Make sure to fill the receipt form as soon as possible after the event (it is 3 days after the purchase or event but just be sure to return the receipt as soon as you can). Lastly, You may have the fill out multiple receipt forms for each receipt you have attained while using that P-Card. 
\end{itemize}

If you are getting pizza, you can always make the order days or hours in advanced and just tell them where and when to bring the pizza. You can also charge tip beforehand as well. For other purchases it it simple, just take the P-Card, use it to make purchases, and then return it.

\subsection{Tutoring}
Tutoring is a very important part of SPS. This is our main income. We are paid \$500 per semester for tutoring. This bill can be found under the Documents tab on the SPS page on wildcatlink (SPS Invoice.docx). To submit the bill, simply modify the necessary name(s) and dates/year and bring the form to Katie. 


\section{Weekly Meetings}
The organization should have regular weekly meetings. The time should be set but the President after figuring out what works best. Nights can often work well as it will reduce class conflicts. 

\subsection{Meeting Notes}
With weekly meetings should be meeting notes that should be sent out in an email so that what is discussed at the meetings will be well documented (and thereby synthesized) and so that members who miss meetings or cannot make meetings can still participate. Some members cannot make meetings, but still deserve to be kept in the loop!

Meeting notes should be taken or synthesized from the meeting. One way of doing this is taking a picture of anything recorded on the board during the meeting and writing up what was discussed with that information. The goal is to organize what was talked about. This is very helpful when it comes to progress as this turns discussion into plans and that will allow plans to become action.

\subsection{Running Meetings}
The meetings should be run by the President, but anyone can help out. This should be discussed amongst the officers. There is no right way to run a meeting. It is important to make sure that any topics previously discussed are brought back to attention. Picking dates and times for events is important as well as coming up with ideas for events. Hopefully meetings can be run as a discussion; however, it will depend on the meeting. It may be helpful to make notes on the board while discussions are happening. Trying to record everyone's ideas is easier than trying to remember them all. It is important that the person who is leading the meeting tries to facilitate conversation rather than control it. It is often good to have discussion and to allow the group to go on tangents. It is the responsibility of the person who is running the meeting to bring the group back to the important topics; however, the org is about socializing, so it should not be discouraged.



\section{Leadership}
This section will outline the responsibilities of each leadership role and how leadership should be transitioned. This should coincide with the constitution. Either this document or the constitution should be modified if they do not coincide.

\subsection{Transitioning Leadership}
It is important that new leaders and primary contacts are changed on the wildcatlink page. Wildcatlink is one of the most official aspects of an org. 

\subsection{President}
Most things

\subsection{Vice President}
some things

\subsection{Treasurer}
N/A

\subsection{Secretary}
N/A


\end{document}



